\documentclass[12pt,fleqn]{examtst}
\usepackage{graphicx}
\usepackage{amssymb}
\usepackage{amsmath}
\usepackage{listings}
\usepackage{multirow}
\usepackage{multicol}
\usepackage{hhline}
\usepackage{booktabs}
\usepackage{url}
\usepackage{enumerate}
\usepackage{hyperref}
%% Comments

\usepackage{color}

\newif\ifcomments\commentstrue

\ifcomments
\newcommand{\authornote}[3]{\textcolor{#1}{[#3 ---#2]}}
\newcommand{\todo}[1]{\textcolor{red}{[TODO: #1]}}
\else
\newcommand{\authornote}[3]{}
\newcommand{\todo}[1]{}
\fi

\newcommand{\wss}[1]{\authornote{blue}{SS}{#1}}

\begin{document}

\newcommand{\soln}{n} %y for yes and n for no

\lstset{language=python, basicstyle=\ttfamily, breaklines=true,
  showspaces=false, showstringspaces=false, breakatwhitespace=true, texcl=true,
  escapeinside={\%*}{*)}}

\newcommand{\codeit}[1]{\texttt{\textit{#1}}}

\begin{center}
  {\large \bf COMP SCI 2ME3 and SFWR ENG 2AA4 Final Examination}\\[1ex]
  {\large \bf McMaster University}\\[1ex]
  \ifthenelse{\equal{\soln}{y}}{\large {\bf Answer Key:} Large arrow
    ($\Longleftarrow$) for correct% , small ($\leftarrow$) for partially
    % correct
  }{}
\end{center}

\medskip

\noindent
DAY CLASS, \textbf{Version 1}  \hfill Dr.~S.~Smith \\
DURATION OF EXAMINATION: 2.5 hours (+ 30 minutes buffer time)\\
MCMASTER UNIVERSITY FINAL EXAMINATION \hfill April 28, 2021

\medskip

\noindent
\rule[3 mm]{\textwidth}{0.5mm}

%\begin{minipage}[t]{1.0\textwidth}

NAME: \wss{Rizwan Ahsan}\\[1ex]

Student ID: \wss{400232878} \\[2mm]

\noindent
\rule[3 mm]{\textwidth}{0.5mm}

This examination paper includes \noofpages pages and
8 % VARIABILITY
questions. You are responsible for ensuring that your copy of the examination
paper is complete. Bring any discrepancy to the attention
of your instructor.\\

\noindent
\emph{By submitting this work, I certify that the work represents solely my own
independent efforts. I confirm that I am expected to exhibit honesty and use
ethical behaviour in all aspects of the learning process.  I confirm that it is
my responsibility to understand what constitutes academic dishonesty under the
\href{https://secretariat.mcmaster.ca/app/uploads/Academic-Integrity-Policy-1-1.pdf}
{Academic Integrity Policy}}.\\

\noindent
\textbf{Special Instructions}:

\begin{enumerate}

\item For taking tests remotely: 
\begin{itemize}
\item Turn off all unnecessary programs, especially Netflix, YouTube, games like
  Xbox or PS4, anything that might be downloading or streaming.
\item If your house is shared, ask others to refrain from doing those activities
  during the test.
\item If you can, connect to the internet via a wired connection.
\item Move close to the Wi-Fi hub in your house. 
\item Restart your computer, 1-2 hours before the exam. A restart can be very
  helpful for several computer hiccups.
\item Use a VPN (Virtual Private Network) since this improves the connection to
  the CAS servers.
\item Commit and push your tex file, compiled pdf file, and code files
  frequently.  As a minimum you should do a commit and push after completing
  each question.
\item Ensure that you push your solution (tex file, pdf file and code files)
  before time expires on the test.  The solution that is in the repo at the
  deadline is the solution that will be graded.
\item If you have trouble with your git repo, the quickest solution may be to
  create a fresh clone.
\end{itemize}
\item It is your responsibility to ensure that the answer sheet is properly
  completed. Your examination result depends upon proper attention to the
  instructions.
\item All physical external resources are permitted, including textbooks, calculators,
  computers, compilers, and the internet.
\item The work has to be completed individually.  Discussion with others is
  strictly prohibited.
\item Read each question carefully.
\item Try to allocate your time sensibly and divide it appropriately between the
  questions.  Use the allocated marks as a guide on how to divide your time
  between questions.
\item The quality of written answers will be considered during grading.  Please
  make your answers well-written and succinct.
\item The set $\mathbb{N}$ is assumed to include $0$.
\end{enumerate}
%\end{minipage}\\

\examheader{CS2ME3/SE2AA4 \ifthenelse{\equal{\soln}{y}} {\hfill SOLUTIONS} }

\renewcommand{\labelenumi}{\Alph{enumi}.}

\newpage

%%%%%%%%%%%%%%%%%%%%%%%%%%%%%%%%%%%%%%%%%%%%%%%%%%%%%%%%%%%%%%%%%%%%%%

\question{5 marks} What are the problems with using ``average lines of code
written per day'' as a metric for programmer productivity?

\bigskip

\noindent \wss{Provide your reasons in the itemized list below.  Add more items
  as required.}

\begin{itemize}
	\item The possibility of the code getting very large is a problem. This will further create the problem of maintaining the code.
	\item Testing all of the cases for the code will be a huge problem. More lines the code has, means the possibility of it having bugs increases.
	\item It takes time to write more lines of code and wasting time to build a product allows errors in the development process because of the deadline of the final product.
	\item It will allow bad writing of code and performance issues since programmers will now be focused on writing lengthy code instead on focusing on to make the code efficient.
	\item This discourages the use of design patterns as there will be more files than required and functions that don't do much to achieve the requirements specification.
	\item Maintanence of the code becomes an issue as we need to document a lot of code. Also the reusability of code in future developments will be very poor.
\end{itemize}

%%%%%%%%%%%%%%%%%%%%%%%%%%%%%%%%%%%%%%%%%%%%%%%%%%%%%%%%%%%%%%%%%%%%%%

\newpage

\question{5 marks} Critique the following requirements specification
for a new cell phone application, called CellApp.  Use the following criteria
discussed in class for judging the quality of the specification: abstract,
unambiguous, and validatable.  How could you improve the requirements
specification?

\bigskip

``The user shall find CellApp easy to use.''

\bigskip

\noindent \wss{Fill in the itemized list below with your answers.  Leave the
  word in bold at the beginning of each item.}

\begin{itemize}
\item \textbf{Abstract} - Yes, the specification is abstract because it doesn't define anything about how the cell phone application needs to be made. It doesn't define how the app should be made to make it easy to use.
\item \textbf{Unambiguous} - The specification is not unambiguous because "easy to use" holds a different meaning for different people. What an app is easy to use differs from different age groups. It doesn't define who the users of the app will be.
\item \textbf{Validatable} - No, the specification is not validatable because it is ambiguous. We cannot validate if the requirements for the app is not unambiguous. We need to define what easy to use means more specifically to validate it.
\item \textbf{How to improve} - The specification should include how the application should be easy to use. Like it should provide a metric for what its means for it being easy to use and what it means to be hard to use. It also should provide who the users of the application will be.
\end{itemize}

%%%%%%%%%%%%%%%%%%%%%%%%%%%%%%%%%%

\newpage

\question{5 marks} The following module is proposed for the maze tracing robot
we discussed in class (L20).  This module is a leaf module in the decomposition
by secrets hierarchy.

\begin{description}
\item [Module Name] find\_path
\item [Module Secret] The data structure and algorithm for finding the shortest
  path in a graph.
\end{description}

\noindent \wss{Fill in the answers to the questions below.  For each item you
  should leave the bold question and write your answer directly after it.}

\begin{enumerate}
\item \textbf{Is this module Hardware Hiding, Software Decision Hiding or
    Behaviour Hiding?  Why?}

  - It is Software Decision Hiding because this module hides what data structures and algorithms are used to find the shortest path.

\item \textbf{Is this a good secret?  Why?}

  - No, it is not a good secret because it holds more than one secret. Normally we would want one module to contain only one secret at a time. So, this module might contain secret for the data structure and another module should contain the secret for the algorithm for finding the shortest path in a graph.

\item \textbf{Does the specification for maze tracing robot require environment
    variables?  If so, which environment variables are needed?}

  - Yes, the specification does require environment variables. It should have an environment variable to determine the width of each cell and also should have a variable to determine the maximum size of the maze.

\end{enumerate}

%%%%%%%%%%%%%%%%%%%%%%%%%%%%%%%%%%

\newpage

\question{5 marks} Answer the following questions assuming that you are in doing
your final year capstone in a group of 5 students.  Your project is to write a
video game for playing chess, either over the network between two human
opponents, or locally between a human and an Artificial Intelligence (AI)
opponent.

\bigskip

\noindent \wss{Fill in the answers to the questions below.  For each item you
  should leave the bold question and write your answer directly after it.}

\begin{enumerate}
  
\item \textbf{You have 8 months to work on the project.  Keeping in mind that we
  usually need to fake a rational design process, what major milestones and what
  timeline for achieving these milestones do you propose?  You can indicate the
  time a milestone is reached by the number of months from the project's start date.}

  - 2nd month: Come with the documentation and the MIS for the game.\\
  4th month: Get a working copy of the game.\\
  6th month: Refine the documentation and MIS.\\
  8th month: Do rigorous testing and fix errors in the game.
  
\item \textbf{Everything in your process should be verified, including the
    verification.  How might you verify your verification?}

  - We can verify our verification by using Mutation Tesing. We can introduce intentional errors in our code by changing various statements, constants, changing the order of execution, etc and use our test cases to see if the same number of errors show or not. 
  
\item \textbf{How do you propose verifying the installability of your game?}

  - We should try to install a fresh copy of the game in a completely new computer other than the ones being used by the 5 students. If we don't have any other computers, we can try to install it in a Virtual Machine. This will allow us to identify what additional software the game might require and verify its installability.
  
\end{enumerate}

%%%%%%%%%%%%%%%%%%%%%%%%%%%%%%%%%%%%%%%%%%%%%%%%%%%%%%%%%%%%%%%%%%%%%%

\newpage

\question{5 marks} As for the previous question, assume you are doing a final
year capstone project in a group of 5 students.  As above, your project
is to write a video game for playing chess, either over the network between two
human opponents, or locally between a human and an Artificial Intelligence
(AI) opponent.  The questions below focus on verification and testing.

\bigskip

\noindent \wss{Fill in the answers to the questions below.  For each item you
  should leave the bold question and right your answer directly after it.}

\begin{enumerate}
\item \textbf{Assume you have 4 work weeks (a work week is 5 days) over the
    course of the project for verification activities.  How many collective
    hours do you estimate that your team has available for verification related
    activities?  Please justify your answer.}

  - Our team will have around collectively 250 hours for verification related activities. We will have test cases for each module that we have written and verify that the tests we have written are correct or not. This will allow us enough time to find and fix any errors and refine the MIS and documentation for the code.
  
\item \textbf{Given the estimated hours available for verification, what verification
    techniques do you recommend for your team?  Please list the techniques,
    along with the number of hours your team will spend on each technique, and
    the reason for selecting this technique.}

  - White Box Testing: We should write test looking at our code to cover all cases from our code. Spend around 50 hours on this.\\
  Black Box Testing: We should write tests looking at the specification to cover any cases not considered from white box testing. Spend around 50 hours on this.\\
  Mutation Testing: We should verify that the test cases we have written are correct or not. This will provide confidence in the code that we have written. Spend around 50 hours on this.\\

  
\item \textbf{Is the oracle problem a concern for implementing your game?  Why
    or why not?  If it is a concern, how do you recommend testing your software?}

  - Yes, the oracle problem is a concern because there is no feasible way to test what the right answer for the case os using Artificial Intelligence should be during a chess match in a local environment. We can set some approximate values that within that treshhold approximates the oracle.
    
\end{enumerate}

%%%%%%%%%%%%%%%%%%%%%%%%%%%%%%%%%%%%%%%%%%%%%%%%%%%%%%%%%%%%%%%%%%%%%%

\newpage

\question{5 marks} Consider the following natural language specification for a
function that looks for resonance when the input matches an integer multiple of
the wavelengths 5 and 7. Provided an integer input between 1 and 1000, the
function returns a string as specified below:

\begin{itemize}
\item If the number is a multiple of 5, then the output is “resonance 5”
\item If the number is a multiple of 7, then the output is “resonance 7”
\item If the number is a multiple of both 5 and 7, then the output is “resonance
  5 and 7”
\item Otherwise, the output is “no resonance”
\end{itemize}

You can assume that inputs outside of the range 1 to 1000 do not occur.

\begin{enumerate}
\item What are the sets $D_i$ that partition $D$ (the input domain) into a
  reasonable set of equivalence classes?

  \wss{answer here - you can answer in natural language, or using mathematical
    notation.}

\item Given the sets $D_i$, and the heuristics discussed in class, how would you
  go about selecting test cases?

  \wss{answer here - you don't need specific test cases; your answer should
    characterize how all significant test cases are to be chosen.}
  
\end{enumerate}
  
%%%%%%%%%%%%%%%%%%%%%%%%%%%%%%%%%%

\newpage

\question{5 marks} Below is a partial specification for an MIS for the game of
tic-tac-toe (\url{https://en.wikipedia.org/wiki/Tic-tac-toe}).  You should
complete the specification.

\bigskip

\wss{The parts that you need to fill in are marked by comments, like this one.
  You can use the given local functions to complete the missing specifications.
  You should not have to add any new local functions, but you can if you feel it
  is necessary for your solution.  As you edit the tex source, please leave the
  \texttt{wss} comments in the file.  You can put your answer immediately
  following the comment.}

\subsection* {Syntax}

\subsubsection* {Exported Constants}

SIZE = 3 {\it //size of the board in each direction}\\

\subsubsection* {Exported Types}

cellT = \{ X, O, FREE \} \\

\subsubsection* {Exported Access Programs}

\begin{tabular}{| l | l | l | p{7cm} |}
\hline
\textbf{Routine name} & \textbf{In} & \textbf{Out} & \textbf{Exceptions}\\
\hline
init & ~ & ~ & ~\\
\hline
move & $\mathbb{N}$, $\mathbb{N}$ & ~ & OutOfBoundsException, InvalidMoveException\\
\hline
getb & $\mathbb{N}$, $\mathbb{N}$ & cellT & OutOfBoundsException\\
\hline
get\_turn & ~ & cellT & ~\\
\hline
is\_valid\_move & $\mathbb{N}$, $\mathbb{N}$ & $\mathbb{B}$ & OutOfBoundsException\\
\hline
is\_winner & cellT & $\mathbb{B}$ & ~\\
\hline
is\_game\_over & ~ & $\mathbb{B}$ & ~\\
\hline

\end{tabular}

\subsection* {Semantics}

\subsubsection* {State Variables}

$b$: boardT\\
$\mathit{Xturn}$: $\mathbb{B}$

\subsubsection* {State Invariant}

\wss{Place your state invariant or invariants here}\\

\subsubsection* {Assumptions}

The init method is called for the abstract object before any other access routine is called for that
object.  The init method can be used to return the state of the game to the state of a new game.

\subsubsection* {Access Routine Semantics}

init():
\begin{itemize}
\item transition: 
$$\mathit{Xturn}, b := \text{true}, 
< \begin{array}{c}
< \mbox{FREE}, \mbox{FREE}, \mbox{FREE} >\\
< \mbox{FREE}, \mbox{FREE}, \mbox{FREE} >\\
< \mbox{FREE}, \mbox{FREE}, \mbox{FREE} >\\
\end{array} >
$$
\item exception: none
\end{itemize}

\noindent move($i$, $j$):
\begin{itemize}
\item transition: $\mathit{Xturn}, b[i, j] := \neg \mathit{Xturn}, (\mathit{Xturn} \Rightarrow \mbox{X} | \neg
\mathit{Xturn} \Rightarrow \mbox{O})$
\item exception
$$exc := (\mbox{InvalidPosition}(i, j) \Rightarrow \mbox{OutOfBoundsException} | \neg \mbox{is\_valid\_move}(i, j)
\Rightarrow \mbox{InvalidMoveException})$$
\end{itemize}

\noindent getb(i, j):
\begin{itemize}
\item output: $\mathit{out} := b[i, j]$
\item exception
$exc := (\mbox{InvalidPosition}(i, j) \Rightarrow \mbox{OutOfBoundsException})$
\end{itemize}

\noindent get\_turn():
\begin{itemize}
\item output: \wss{Return the cellT that corresponds to the current turn}\\

\item exception: none
\end{itemize}

\noindent is\_valid\_move(i, j):
\begin{itemize}
\item output: $\mathit{out} := (b[i][j] = \mbox{FREE})$ 
\item exception $exc := (\mbox{InvalidPosition}(i, j) \Rightarrow \mbox{OutOfBoundsException})$
\end{itemize}

\noindent is\_winner(c):
\begin{itemize}
\item output: $\mathit{out} := \mbox{horizontal\_win}(c, b) \vee \mbox{vertical\_win}(c, b) \vee
\mbox{diagonal\_win}(c, b)$ 
\item exception: none
\end{itemize}

\noindent is\_game\_over():
\begin{itemize}
\item output: \wss{Returns true if X or O wins, or if there are no more moves remaining}\\

\item exception: none
\end{itemize}

\subsubsection* {Local Types}

boardT = sequence [SIZE, SIZE] of cellT

\subsubsection* {Local Functions}

\noindent \textbf{InvalidPosition}: $\mathbb{N}$ $\times$ $\mathbb{N}$ $\rightarrow$ $\mathbb{B}$\\
~\newline
InvalidPosition$(i, j) \equiv \neg ( ( 0 \leq i < \mbox{SIZE} ) \wedge ( 0 \leq j < \mbox{SIZE}))$

~\newline

\noindent \textbf{count}: cellT $\rightarrow$ $\mathbb{N}$\\
~\newline
\wss{For the current board return the number of occurrences of the cellT
  argument}
~\newline


~\newline

\noindent \textbf{horizontal\_win} : cellT $\times$ boardT $\rightarrow$ $\mathbb{B}$\\
~\newline
horizontal\_win$(c, b) \equiv \exists (i : \mathbb{N} | 0 \leq i < \mbox{SIZE} : b[i, 0] = b[i, 1] = b[i, 2] = c)$

~\newline

\noindent \textbf{vertical\_win} : cellT $\times$ boardT $\rightarrow$ $\mathbb{B}$\\
~\newline
vertical\_win$(c, b) \equiv \exists (j : \mathbb{N} | 0 \leq j < \mbox{SIZE} : b[0, j] = b[1, j] = b[2, j] = c)$

~\newline

\noindent \textbf{diagonal\_win} : cellT $\times$ boardT $\rightarrow$ $\mathbb{B}$\\
~\newline
\wss{Returns true if one of the diagonals for the board has all of the entries
  equal to cellT}
~\newline


%%%%%%%%%%%%%%%%%%%%%%%%%%%%%%%%%%

\newpage

\question{5 marks} For this question you will implement in Java an ADT for a 1D
sequence of real numbers.  We want to take the mean of the numbers in the
sequence, but as the following web-page shows, there are several different
algorithms for doing this: \url{https://en.wikipedia.org/wiki/Generalized_mean}

Given that there are different options, we will use the strategy design pattern,
as illustrated in the following UML diagram:

\begin{figure}[!h]
\begin{center}
\includegraphics[scale=0.7]{Seq1D_Mean_Strategy_UML.png}
\end{center}
\caption{UML Class Diagram for Seq1D with Mean Function, using Strategy
  Pattern} \label{Fig_UML_Strategy}
\end{figure}

You will need to fill in the following blank files:
\texttt{MeanCalculator.java}, \texttt{HarmonicMean.java},
\texttt{QuadraticMean.java}, and \texttt{Seq1D.java}.  Two testing files are
also provided: \texttt{Expt.java} and \texttt{TestSeq1D.java}.  The file
\texttt{Expt.java} is pre-populated with some simple experiments to help you see
the interface in use, and do some initial testing.  You are free to add to this
file to experiment with your work, but the file itself isn't graded.  The
\texttt{TestSeq1D.java} is also not graded.  However, you may want to create
test cases to improve your confidence in your solution.  The stubs of the
necessary files are already available in your \texttt{src} folder.  The code
will automatically be imported into this document when the \texttt{tex} file is
compiled.  You should use the provided Makefile to test your code.  You will NOT
need to modify the Makefile.  The given Makefile will work for \texttt{make
  test}, without errors, from the initial state of your repo.  The \texttt{make
  expt} rule will also work, because all lines of code have been commented out.
Uncomment lines as you complete work on each part of the modules relevant to
those lines in \texttt{Expt.java} file.  As usual, the final test is whether the
code runs on mills.  You do not need to worry about doxygen comments.

Any exceptions in the specification have names identical to the expected Java
exceptions; your code should use exactly the exception names as given in the
spec.

Remember, your code needs to implement the given specification so that the
interface behaves as specified.  This does NOT mean that the local functions
need to all be implemented, or that the types used internally to the spec need
to be implemented exactly as given.  If you do implement any local functions,
please make them private.  The real type in the MIS should be implemented by
\texttt{Double} (capital D) in Java.

\wss{Complete Java code to match the following specification.}

%%%%%%%%%%%%%%%%%%%%%%%%%%%%%%%%%%

\newpage

\section* {Mean Calculator Interface Module}

\subsection*{Interface Module}

MeanCalculator

\subsection* {Uses}

None

\subsection* {Syntax}

\subsubsection* {Exported Constants}

None

\subsubsection* {Exported Types}

None 

\subsubsection* {Exported Access Programs}

\begin{tabular}{| l | l | l | p{5cm} |}
\hline
\textbf{Routine name} & \textbf{In} & \textbf{Out} & \textbf{Exceptions}\\
\hline
meanCalc & seq of $\mathbb{R}$ & $\mathbb{R}$ & ~\\
\hline
\end{tabular}

\subsubsection* {Considerations}

meanCalc calculates the mean (a real value) from a given sequence of reals.
The order of the entries in the sequence does not matter.

%%%%%%%%%%%%%%%%%%%%%%%%%%%%%%%%%%

\newpage

\section* {Harmonic Mean Calculation}

\subsection*{Template Module inherits MeanCalculator}

HarmonicMean

\subsection* {Uses}

MeanCalculator

\subsection* {Syntax}

\subsubsection* {Exported Constants}

None

\subsubsection* {Exported Types}

None 

\subsubsection* {Exported Access Programs}

\begin{tabular}{| l | l | l | p{5cm} |}
\hline
\textbf{Routine name} & \textbf{In} & \textbf{Out} & \textbf{Exceptions}\\
\hline
meanCalc & seq of $\mathbb{R}$ & $\mathbb{R}$ & ~\\
\hline
\end{tabular}

\subsection* {Semantics}

\subsubsection* {State Variables}

None

\subsubsection* {State Invariant}

None

\subsubsection* {Assumptions}

None

\subsubsection* {Access Routine Semantics}

meanCalc($v$)
\begin{itemize}
\item output: $\mathit{out} := \frac{|x|}{+(x: \mathbb{R} | x \in v : 1/x)}$
\item exception: none
\end{itemize}

%%%%%%%%%%%%%%%%%%%%%%%%%%%%%%%%%%

\newpage

\section* {Quadratic Mean Calculation}

\subsection*{Template Module inherits MeanCalculator}

QuadraticMean

\subsection* {Uses}

MeanCalculator

\subsection* {Syntax}

\subsubsection* {Exported Constants}

None

\subsubsection* {Exported Types}

None 

\subsubsection* {Exported Access Programs}

\begin{tabular}{| l | l | l | p{5cm} |}
\hline
\textbf{Routine name} & \textbf{In} & \textbf{Out} & \textbf{Exceptions}\\
\hline
meanCalc & seq of $\mathbb{R}$ & $\mathbb{R}$ & ~\\
\hline
\end{tabular}

\subsection* {Semantics}

\subsubsection* {State Variables}

None

\subsubsection* {State Invariant}

None

\subsubsection* {Assumptions}

None

\subsubsection* {Access Routine Semantics}

meanCalc($v$)
\begin{itemize}
\item output: $\mathit{out} := \sqrt{\frac{+(x: \mathbb{R} | x \in v : x^2)}{|x|}}$
\item exception: none
\end{itemize}

%%%%%%%%%%%%%%%%%%%%%%%%%%%%%%%%%%

\newpage

\section* {Seq1D Module}

\subsection* {Template Module}

Seq1D

\subsection* {Uses}

MeanCalculator

\subsection* {Syntax}

\subsubsection* {Exported Types}

Seq1D = ?

\subsubsection* {Exported Constants}

None

\subsubsection* {Exported Access Programs}

\begin{tabular}{| l | l | l | p{6cm} |}
\hline
\textbf{Routine name} & \textbf{In} & \textbf{Out} & \textbf{Exceptions}\\
\hline
new Seq1D & seq of $\mathbb{R}$, MeanCalculator & Seq1D & IllegalArgumentException\\
\hline
setMaxCalculator & MaxCalculator &  & \\
\hline
mean &  & $\mathbb{R}$ & \\
\hline

\end{tabular}

\subsection* {Semantics}

\subsubsection* {State Variables}

$s$: seq of $\mathbb{R}$\\
meanCalculator: MeanCalculator

\subsubsection* {State Invariant}

None

\subsubsection* {Assumptions}

\begin{itemize}
\item The Seq1D constructor is called for each object instance before any other
  access routine is called for that object.  The constructor can only be called
  once.  All real numbers provided to the constructor will be zero or positive.
\end{itemize}

\subsubsection* {Access Routine Semantics}

new Seq1D($x$, $m$):
\begin{itemize}
\item transition: $s, \text{meanCalculator} := x, m$
\item output: $\mathit{out} := \mathit{self}$
\item exception:
  $\mathit{exc} := (|x| = 0 \Rightarrow \mbox{IllegalArgumentException})$
\end{itemize}

\noindent setMeanCalculator($m$):
\begin{itemize}
\item transition: $\mbox{meanCalculator} := m$
\item exception: none
\end{itemize}

\noindent mean():
\begin{itemize}
\item output: $\mathit{out} := \mbox{meanCalculator.meanCalc}()$
\item exception: none
\end{itemize}

%%%%%%%%%%%%%%%%%%%%%%%%%%%%%%%%%%

\newpage

\subsection*{Code for MeanCalculator.java}

\noindent \lstinputlisting[language = Java]{./src/MeanCalculator.java}

\newpage

\subsection*{Code for HarmonicMean.java}

\noindent \lstinputlisting[language = Java]{./src/HarmonicMean.java}

\newpage

\subsection*{Code for QuadraticMean.java}

\noindent \lstinputlisting[language = Java]{./src/QuadraticMean.java}

\newpage

\subsection*{Code for Seq1D.java}

\noindent \lstinputlisting[language = Java]{./src/Seq1D.java}

%%%%%%%%%%%%%%%%%%%%%%%%%%%%%%%%%%

\end{document}