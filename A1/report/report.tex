\documentclass[12pt]{article}

\usepackage{graphicx}
\usepackage{paralist}
\usepackage{listings}
\usepackage{booktabs}
\usepackage{hyperref}

\oddsidemargin 0mm
\evensidemargin 0mm
\textwidth 160mm
\textheight 200mm

\pagestyle {plain}
\pagenumbering{arabic}

\newcounter{stepnum}

\title{Assignment 1 Solution}
\author{Rizwan Ahsan, ahsanm7}
\date{\today}

\begin {document}

\maketitle

This report discusses testing of the \verb|ComplexT| and \verb|TriangleT|
classes written for Assignment 1. It also discusses testing of the partner's
version of the two classes. The design restrictions for the assignment
are critiqued and then various related discussion questions are answered.

\section{Assumptions and Exceptions} \label{AssumptAndExcept}
The following assumtions were made:
\begin{itemize}
	\item The arguments provided for the constructor of \verb|ComplexT| class will always be of 
	type float.
	\item The arguments provided for the constructor of \verb|TriangleT| class will always be of type integer and their values will be greater than 0.
\end{itemize}
The following exceptions were handled:
\begin{itemize}
	\item The \verb|recip| method will throw an exception when the value of the denominator variable calculated from the object's real and imaginary numbers is less than or equal 
	to 0 i.e \(x^2 + y^2 <= 0\), x being the real part and y being the imaginary part.
	\item The \verb|div| method will throw a similar exception to \verb|recip| but in this case
	the exception is thrown for the passed in argument's real and imaginary numbers instead.
\end{itemize}


\section{Test Cases and Rationale} \label{Testing}


\section{Results of Testing Partner's Code}


\section{Critique of Given Design Specification}


\section{Answers to Questions}

\begin{enumerate}[(a)]

\item 
\item ...

\end{enumerate}

\newpage

\lstset{language=Python, basicstyle=\tiny, breaklines=true, showspaces=false,
  showstringspaces=false, breakatwhitespace=true}
%\lstset{language=C,linewidth=.94\textwidth,xleftmargin=1.1cm}

\def\thesection{\Alph{section}}

\section{Code for complex\_adt.py}

\noindent \lstinputlisting{../src/complex_adt.py}

\newpage

\section{Code for triangle\_adt.py}

\noindent \lstinputlisting{../src/triangle_adt.py}

\newpage

\section{Code for test\_driver.py}

\noindent \lstinputlisting{../src/test_driver.py}

\newpage

\section{Code for Partner's complex\_adt.py}

\noindent \lstinputlisting{../partner/complex_adt.py}

\section{Code for Partner's triangle\_adt.py}

\noindent \lstinputlisting{../partner/triangle_adt.py}

\end {document}